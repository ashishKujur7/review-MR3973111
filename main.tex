\documentclass[12pt]{amsart}
\usepackage[margin=1in]{geometry}
\usepackage{amsfonts, amsmath}
\usepackage[T1]{fontenc}
\usepackage{mathrsfs, enumitem}
\usepackage{hyperref}
\usepackage[utf8]{inputenc}
\usepackage{amssymb}
\usepackage{amsfonts}
\usepackage{amsmath}
\usepackage{amsthm,cancel}
\usepackage{color}
\usepackage{hyperref}
\usepackage{csquotes}
%\usepackage{cmbright}
\usepackage{tikz-cd}
\usepackage{lipsum}
\usepackage[maxnames=999,backend=biber,style=alphabetic]{biblatex}
\addbibresource{ref.bib}
%\usepackage{stmaryrd}

\newtheorem{theorem}{Theorem}[section]
\newtheorem{lemma}[theorem]{Lemma}
\newtheorem{claim}[theorem]{Claim}
\newtheorem{proposition}[theorem]{Proposition}
\newtheorem{corollary}[theorem]{Corollary}
\newtheorem{fact}[theorem]{Fact}
\newtheorem{notation}[theorem]{Notation}
\newtheorem{observation}[theorem]{Observation}
\newtheorem{conjecture}[theorem]{Conjecture}
\newtheorem{exercise}[theorem]{Exercise}
\newtheorem{question}[theorem]{Question}

\theoremstyle{definition}
\newtheorem{definition}[theorem]{Definition}
\newtheorem{example}[theorem]{Example}
\numberwithin{equation}{section}

\theoremstyle{remark}
\newtheorem{remark}[theorem]{Remark}
\theoremstyle{plain}
\newcommand{\ignore}[1]{}

% section symbol
%\renewcommand{\thesection}{\S\arabic{section}}

% \renewcommand{\Pr}{{\bf Pr}}
% \newcommand{\Prx}{\mathop{\bf Pr\/}}
% \newcommand{\E}{{\bf E}}
% \newcommand{\Ex}{\mathop{\bf E\/}}
% \newcommand{\Var}{{\bf Var}}
% \newcommand{\Varx}{\mathop{\bf Var\/}}
% \newcommand{\Cov}{{\bf Cov}}
% \newcommand{\Covx}{\mathop{\bf Cov\/}}

% shortcuts for symbol names that are too long to type
\newcommand{\eps}{\epsilon}
\newcommand{\lam}{\lambda}
\renewcommand{\l}{\ell}
\newcommand{\la}{\langle}
\newcommand{\ra}{\rangle}
\newcommand{\wh}{\widehat}
\newcommand{\wt}{\widetilde}

% % "blackboard-fonted" letters for the reals, naturals etc.
\newcommand{\R}{\mathbb R}
\newcommand{\N}{\mathbb N}
\newcommand{\Z}{\mathbb Z}
\newcommand{\F}{\mathbb F}
\newcommand{\Q}{\mathbb Q}
\newcommand{\C}{\mathbb C}
\newcommand{\D}{\mathbb D}
\newcommand{\T}{\mathbb T}

% % operators that should be typeset in Roman font
% \newcommand{\poly}{\mathrm{poly}}
% \newcommand{\polylog}{\mathrm{polylog}}
% \newcommand{\sgn}{\mathrm{sgn}}
% \newcommand{\avg}{\mathop{\mathrm{avg}}}
% \newcommand{\val}{{\mathrm{val}}}

% % complexity classes
% \renewcommand{\P}{\mathrm{P}}
% \newcommand{\NP}{\mathrm{NP}}
% \newcommand{\BPP}{\mathrm{BPP}}
% \newcommand{\DTIME}{\mathrm{DTIME}}
% \newcommand{\ZPTIME}{\mathrm{ZPTIME}}
% \newcommand{\BPTIME}{\mathrm{BPTIME}}
% \newcommand{\NTIME}{\mathrm{NTIME}}

% values associated to optimization algorithm instances
\newcommand{\Opt}{{\mathsf{Opt}}}
\newcommand{\Alg}{{\mathsf{Alg}}}
\newcommand{\Lp}{{\mathsf{Lp}}}
\newcommand{\Sdp}{{\mathsf{Sdp}}}
\newcommand{\Exp}{{\mathsf{Exp}}}

% if you think the sum and product signs are too big in your math mode; x convention
% as in the probability operators
\newcommand{\littlesum}{{\textstyle \sum}}
\newcommand{\littlesumx}{\mathop{{\textstyle \sum}}}
\newcommand{\littleprod}{{\textstyle \prod}}
\newcommand{\littleprodx}{\mathop{{\textstyle \prod}}}

% horizontal line across the page
\newcommand{\horz}{
\vspace{-.4in}
\begin{center}
\begin{tabular}{p{\textwidth}}\\
\hline
\end{tabular}
\end{center}
}

% calligraphic letters
\newcommand{\calA}{{\cal A}}
\newcommand{\calB}{{\cal B}}
\newcommand{\calC}{{\cal C}}
\newcommand{\calD}{{\cal D}}
\newcommand{\calE}{{\cal E}}
\newcommand{\calF}{{\cal F}}
\newcommand{\calG}{{\cal G}}
\newcommand{\calH}{{\mathcal H}}
\newcommand{\calI}{{\cal I}}
\newcommand{\calJ}{{\cal J}}
\newcommand{\calK}{{\cal K}}
\newcommand{\calL}{{\cal L}}
\newcommand{\calM}{{\mathcal M}}
\newcommand{\calN}{{\cal N}}
\newcommand{\calO}{{\cal O}}
\newcommand{\calP}{{\cal P}}
\newcommand{\calQ}{{\cal Q}}
\newcommand{\calR}{{\cal R}}
\newcommand{\calS}{{\cal S}}
\newcommand{\calT}{{\cal T}}
\newcommand{\calU}{{\cal U}}
\newcommand{\calV}{{\cal V}}
\newcommand{\calW}{{\cal W}}
\newcommand{\calX}{{\cal X}}
\newcommand{\calY}{{\cal Y}}
\newcommand{\calZ}{{\cal Z}}

% bold letters (useful for random variables)
%----------------------------------------------
% \renewcommand{\a}{{\boldsymbol a}}
% \renewcommand{\b}{{\boldsymbol b}}
% \renewcommand{\c}{{\boldsymbol c}}
% \renewcommand{\d}{{\boldsymbol d}}
% \newcommand{\e}{{\boldsymbol e}}
% \newcommand{\f}{{\boldsymbol f}}
% \newcommand{\g}{{\boldsymbol g}}
% \newcommand{\h}{{\boldsymbol h}}
% \renewcommand{\i}{{\boldsymbol i}}
% \renewcommand{\j}{{\boldsymbol j}}
% \renewcommand{\k}{{\boldsymbol k}}
% \newcommand{\m}{{\boldsymbol m}}
% \newcommand{\n}{{\boldsymbol n}}
% \renewcommand{\o}{{\boldsymbol o}}
% \newcommand{\p}{{\boldsymbol p}}
% \newcommand{\q}{{\boldsymbol q}}
% \renewcommand{\r}{{\boldsymbol r}}
% \newcommand{\s}{{\boldsymbol s}}
% \renewcommand{\t}{{\boldsymbol t}}
% \renewcommand{\u}{{\boldsymbol u}}
% \renewcommand{\v}{{\boldsymbol v}}
% \newcommand{\w}{{\boldsymbol w}}
% \newcommand{\x}{{\boldsymbol x}}
% \newcommand{\y}{{\boldsymbol y}}
% \newcommand{\z}{{\boldsymbol z}}
% \newcommand{\A}{{\boldsymbol A}}
% \newcommand{\B}{{\boldsymbol B}}
% \newcommand{\C}{{\boldsymbol C}}
% \newcommand{\D}{{\boldsymbol D}}
% \newcommand{\E}{{\boldsymbol E}}
% \newcommand{\F}{{\boldsymbol F}}
% \newcommand{\G}{{\boldsymbol G}}
% \renewcommand{\H}{{\boldsymbol H}}
% \newcommand{\I}{{\boldsymbol I}}
% \newcommand{\J}{{\boldsymbol J}}
% \newcommand{\K}{{\boldsymbol K}}
% \renewcommand{\L}{{\boldsymbol L}}
% \newcommand{\M}{{\boldsymbol M}}
% \renewcommand{\O}{{\boldsymbol O}}
% \renewcommand{\P}{{\mathbb{P}}}
% \newcommand{\Q}{{\boldsymbol Q}}
% \newcommand{\R}{{\boldsymbol R}}
% \renewcommand{\S}{{\boldsymbol S}}
% \newcommand{\T}{{\boldsymbol T}}
% \newcommand{\U}{{\boldsymbol U}}
% \newcommand{\V}{{\boldsymbol V}}
% \newcommand{\W}{{\boldsymbol W}}
% \newcommand{\X}{{\boldsymbol X}}
% \newcommand{\Y}{{\boldsymbol Y}}
% \newcommand{\Z}{{\boldsymbol Z}}

% script letters
\newcommand{\scrA}{{\mathscr A}}
\newcommand{\scrB}{{\mathscr B}}
\newcommand{\scrC}{{\mathscr C}}
\newcommand{\scrD}{{\mathscr D}}
\newcommand{\scrE}{{\mathscr E}}
\newcommand{\scrF}{{\mathscr F}}
\newcommand{\scrG}{{\mathscr G}}
\newcommand{\scrH}{{\mathscr H}}
\newcommand{\scrI}{{\mathscr I}}
\newcommand{\scrJ}{{\mathscr J}}
\newcommand{\scrK}{{\mathscr K}}
\newcommand{\scrL}{{\mathscr L}}
\newcommand{\scrM}{{\mathscr M}}
\newcommand{\scrN}{{\mathscr N}}
\newcommand{\scrO}{{\mathscr O}}
\newcommand{\scrP}{{\mathscr P}}
\newcommand{\scrQ}{{\mathscr Q}}
\newcommand{\scrR}{{\mathscr R}}
\newcommand{\scrS}{{\mathscr S}}
\newcommand{\scrT}{{\mathscr T}}
\newcommand{\scrU}{{\mathscr U}}
\newcommand{\scrV}{{\mathscr V}}
\newcommand{\scrW}{{\mathscr W}}
\newcommand{\scrX}{{\mathscr X}}
\newcommand{\scrY}{{\mathscr Y}}
\newcommand{\scrZ}{{\mathscr Z}}

\newcommand{\im}{{\text{im }}}
\newcommand{\ip}[1]{\left\langle #1 \right\rangle}
\newcommand{\norm}[1]{\left\lVert #1 \right\rVert}
\newcommand{\abs}[1]{\left\lvert #1 \right\rvert}
\newcommand{\defbox}[1]{\fbox{\textsc{ #1 }}}
\newcommand{\coker}{\operatorname{coker}}
\newcommand{\ind}{\operatorname{ind}}
\newcommand{\supp}{\operatorname{supp}}
\newcommand{\dou}{\partial}
\newcommand{\Mult}{\operatorname{Mult}}

%\renewcommand*{\descriptionlabel}[1]{\hspace{\labelsep}\descfont #1:}

\newcommand\blfootnote[1]{%
  \begingroup
  \renewcommand\thefootnote{}\footnote{#1}%
  \addtocounter{footnote}{-1}%
  \endgroup
}

\title{A Review of Interpolating Sequences in Spaces with the Complete Pick Property}
\author{Ashish Kujur}
\date{\today}

\begin{document}
\maketitle
In this paper under review titled \citetitle{zbMATH07130838} (see \cite{zbMATH07130838}), \citeauthor{zbMATH07130838} generalise a result due to \citeauthor{MR0117349} for the Hardy space \cite{MR0117349} and of \citeauthor{Marshall1994InterpolatingSF} for the Dirichlet space \cite{Marshall1994InterpolatingSF}. 

%The setup is as follows: A \textit{reproducing kernel Hilbert space} $\calH$ (rkHs, in short) is a subset of complex valued functions on a set $X$ which is equipped with a Hilbert space structure such that the pointwise evaluations $f \to f(z)$ is bounded for each $z\in X$ and the kernel function $k : X \times X \to \C$ satisfies that $k\left( z,z \right) \ne 0$ for all $z\in X$. Associated with every rkHs $\calH$ is its \textit{multiplier algebra $\calM \left( \calH \right)$} which is the set of all functions $\varphi : X \to \C$ such that $\varphi f \in \calH$ for all $f\in \calH$.

\citeauthor{MR0117349} showed in \cite{MR0117349} that for a complex sequence $\left( \lambda_{i} \right)$ satisfying $\abs{\lambda_{i}} < 1$ for each $i \in \N$, the following are equivalent:
\begin{enumerate}
\item For any bounded complex sequence $\left( w_{i} \right)$, there is $f\in H^{\infty} \left( \D \right)$ (which is the \textit{multiplier algebra} of the Hardy-Hilbert space $H^{2} \left( \D \right)$) such that $f\left( z_{i} \right)=w_{i}$ for each $i\in \N$.
\item There is some $c>0$ such that for each $i \in \N$, 
\begin{equation}
\prod_{j\ne i} \abs{\frac{\lambda_{i}-\lambda_{j}}{1-\bar \lambda_{i} \lambda_{j}}} \ge c > 0.
\label{eqn:phbm}
\end{equation}
\end{enumerate}
In modern terminology, a sequence $\left( \lambda_{i} \right) \subset \D$, the open unit disk, is called \textit{an interpolating sequence for $H^{\infty} \left( \D \right)$} if it satisfies at least one of the condition (hence, both) of \citeauthor{MR0117349}'s result. Intuitively, it means a sequence is interpolating for $H^{\infty} \left( \D \right)$ iff the points are sufficiently spread out in the hyperbolic metric of the open unit disc. This is an conclusion that one can make by observing equation \ref{eqn:phbm}. An analogous result can be formulated in the context of \textit{reproducing kernel Hilbert spaces} (or rkHs, inshort).

\citeauthor{zbMATH07130838} in \cite{zbMATH07130838} show that for a complete Pick space $\calH$, a sequence is interpolating for its multiplier algebra $\calM \left( \calH \right)$ iff it is separated and generates a Carleson measure. More specifically, if $\calH$ is complete Pick space (a special rkHs where positivity of Pick matrix with matricial entries implies interpolation by multipliers, see \cite{MR1882259} for a precise definition) then the following are equivalent:
\begin{description}
\item[(IM)] $\left( \Lambda \right) = \left( \lambda_{i} \right) \subset X$ is \textit{interpolating for $\calM \left( \calH \right)$}, that is, whenever $\left( w_{i} \right)_{i \in \N} \in \ell ^{\infty}$, there is a multiplier $\varphi \in \calM \left( \calH \right)$ such that $\varphi \left( \lambda_{i} \right) = w_{i}$ for each $i$, and,
\item[(S+C)] the following two hold:
\begin{description}
\item[(S)] $\Lambda$ is \textit{separated with the pseudometric $d$} on $X$ given by
\begin{equation*}
d\left( z,w \right) = \sqrt{1- \frac{\abs{k\left( z,w \right)}^{2}}{k\left( z,z \right)k\left( w,w \right)}}  \qquad (z,w \in X) ,
\end{equation*}
that is, there is some $c> 0$ such that $d\left( \lambda_{i}, \lambda_{j} \right) \ge c > 0$ for each $i\ne j$ and
\item[(C)] the atomic measure $\mu$ given by $\displaystyle \mu = \sum_{i}\frac{\delta_{\lambda_{i}}}{k\left( \lambda_{i}, \lambda_{i} \right)}$ is a \textit{Carleson measure for $\calH$}, in other words, there is some $c> 0$ such that 
\begin{equation*}
\sum_{i} \frac{\abs{f\left( \lambda_{i} \right)}^{2}}{k\left( \lambda_{i}, \lambda_{i} \right)} \le c \norm{f}^{2} \qquad \left( f\in \calH \right).
\end{equation*}
\end{description}
\end{description}

This settled a 20 year problem posed by \citeauthor{MR1882259} in Chapter 9 of their monograph \citetitle{MR1882259} \cite{MR1882259}. It is interesting to note that the proof used an equivalent form of the Kadison Singer problem, called the Paving Conjecture (as demonstrated in \cite{33675660-3119-37a8-a5d2-fa5a40dfb227}). The long standing Kadison Singer problem was resolved by \citeauthor{MR3374963} in \cite{MR3374963} in 2013. 

Finally, the authors explore interpolating sequences for multipliers of pairs of reproducing kernel Hilbert spaces. \citeauthor{zbMATH07704844} resolved some of the problems asked by the authors in his paper \cite{zbMATH07704844}.
\nocite{*}
\printbibliography

\end{document}
